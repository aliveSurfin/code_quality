\section{Introduction}
Measuring code quality can be a subjective task as what is quality code can differ depending on the purpose of the code and the criticality of the produced software.
For example the guidance system on an aeroplane will have much higher quality dependencies than a computing students first Java program.
What applies to both of these scenarios is that we expect the code that is written to fulfil it's intended purpose.
The more critical the piece of software the more we must ensure for code quality, some of the things we are looking for are as such.
\begin{itemize}
    \item The code must do what it is meant to do.
    \item The code must be able to be tested.
    \item The code must be well documented.
    \item The code is readable and understandable.
    \item The code must be extendable.
\end{itemize}
Code Quality can be accessed by many methods including manual code review but the problem with code review 
is how long it can take, automatic code analysis tools can speed up and further improve code review. 

In this report we will discuss
\begin{itemize}
    \item The Background of the problem 
    \item The Methodology behind development 
    \item The Development process 
    \item The Created Product
    \item Appraisal of the system and work
    \item Conclusions about the project
    \item Scope for Future work 
\end{itemize}
