\documentclass[sigconf, nonacm]{acmart}
\usepackage{pdfpages}
\usepackage{amsmath}
\usepackage{listings}
\newcommand*\textfrac[2]{
  \frac{\text{#1}}{\text{#2}}
}
\newcommand{\RefFig}[1]{~\ref{#1} on page ~\pageref{#1}}
\usepackage{color}
\definecolor{lightgray}{rgb}{.9,.9,.9}
\definecolor{darkgray}{rgb}{.4,.4,.4}
\definecolor{purple}{rgb}{0.65, 0.12, 0.82}
\lstdefinelanguage{JavaScript}{
  keywords={break, case, catch, continue, debugger, default, delete, do, else, false, finally, for, function, if, in, instanceof, new, null, return, switch, this, throw, true, try, typeof, var, void, while, with},
  morecomment=[l]{//},
  morecomment=[s]{/*}{*/},
  morestring=[b]',
  morestring=[b]",
  ndkeywords={class, export, boolean, throw, implements, import, this},
  keywordstyle=\color{blue}\bfseries,
  ndkeywordstyle=\color{darkgray}\bfseries,
  identifierstyle=\color{black},
  commentstyle=\color{purple}\ttfamily,
  stringstyle=\color{red}\ttfamily,
  sensitive=true
}

\lstset{
   language=JavaScript,
   backgroundcolor=\color{lightgray},
   extendedchars=true,
   basicstyle=\footnotesize\ttfamily,
   showstringspaces=false,
   showspaces=false,
   numbers=left,
   numberstyle=\footnotesize,
   numbersep=9pt,
   tabsize=2,
   breaklines=true,
   showtabs=false,
   captionpos=b
}
%% complete the rights form.
\setcopyright{none}

%% These commands are for a PROCEEDINGS abstract or paper.
\acmConference[Dundee Computing Honours Project '22]{Computing Honours Projects '22: The University of Dundee Computing Project Showcase: '22.}{2022}{Dundee, UK}


%%
%% end of the preamble, start of the body of the document source.
\begin{document}

%%
%% The "title" command has an optional parameter,
%% allowing the author to define a "short title" to be used in page headers.
\title{Code Quality Analysis}


\author{Christy McCarron}
\email{cswmccarron@dundee.ac.uk}
\affiliation{%
  \institution{University of Dundee}
  \city{Dundee}
  \state{Scotland}
  \country{UK}
}

%%
%% The abstract is a short summary of the work to be presented in the
%% article.

\begin{abstract}
Your abstract will go here.
\end{abstract}


%% A "teaser" image appears between the author and affiliation
%% information and the body of the document, and typically spans the
%% page.
\begin{teaserfigure}
  \includegraphics[width=\textwidth]{images/third_way.png}
  \caption{Third Way, xkcd}
  \Description{xkcd comic on language choices}
  \label{fig:thirdWay}
\end{teaserfigure}

%%
%% This command processes the author and affiliation and title
%% information and builds the first part of the formatted document.
\maketitle


% THIS IS THE MAIN PARTS OF THE DOCUMENT
\section{Introduction}
Measuring code quality can be a subjective task as what is quality code can differ depending on the purpose of the code and the criticality of the produced software.
For example the guidance system on an aeroplane will have much higher quality dependencies than a computing students first Java program.
What applies to both of these scenarios is that we expect the code that is written to fulfil it's intended purpose.
The more critical the piece of software the more we must ensure for code quality, some of the things we are looking for are as such.
\begin{itemize}
    \item The code must do what it is meant to do.
    \item The code must be able to be tested.
    \item The code must be well documented.
    \item The code is readable and understandable.
    \item The code must be extendable.
\end{itemize}
Code Quality can be accessed by many methods including manual code review but the problem with code review 
is how long it can take, automatic code analysis tools can speed up and further improve code review. 

In this report we will discuss
\begin{itemize}
    \item The Background of the problem 
    \item The Methodology behind development 
    \item The Development process 
    \item The Created Product
    \item Appraisal of the system and work
    \item Conclusions about the project
    \item Scope for Future work 
\end{itemize}

\section{Background}
Here we will dicuss the background of the problem, where code quality can be used, what are the benefits to using it and how 
can we create a similar tool.
\subsection{Applicability to Industry}
Where we start to move towards ensuring more quality in our written code is when developing code in Industry,
many techniques are utilised which includes the use of code review.
\subsubsection{Code Review}
Manual code review is described as peers reviewing the code that has been written to look for problems within the written code,
the results of this can be staggering on the outcomes of delivered software.
In their 2006 book Jason Cohen describes the outcome of a company after implementing code review for only 3 months. \cite{cohen2006best}
\begin{verbatim}
The result: Code review would have saved half the cost of
fixing the bugs. Plus they would have found 162 
additional bugs.
\end{verbatim}
This shows the incredible power of having another pair of eyes on code before it is shipped,
not only to reduce bugs as previously stated but to transfer knowledge, increase team awareness and create alternative solutions to problems \cite{modernCodeReview}


\subsubsection{Use of tools in Code Review}
As we have shown that code review is an important and effective task for creating quality software, we must understand the limitations it has.
One of the biggest problems with Code Reviews is how long they can take
the use of code quality tools speeds up the code review process by identifying potential problem areas in the code. \cite{confusionInCodeReviews}
\newline
The automatic detection and amending of simple errors in code is said to allow for developers to utilise code review for deeper and more subtle issues. \cite{modernCodeReview}
\begin{verbatim}
Code review is fertile ground to have an impact with 
code analysis tools.
\end{verbatim}\cite{modernCodeReview}




\subsection{Static Analysis}
In this project the student will focus on Static Analysis methods which are performed on source code this
is as opposed to Dynamic Analysis which is the analysis of the properties of a running program \cite{dynamicAnalysis}
\subsubsection{\textbf{Cyclomatic Complexity (CC)}}
is defined as
\begin{verbatim}
The number of linearly independent paths within a 
piece of code
\end{verbatim}
The piece of code in figure \RefFig{fig:simple-func1} has a CC of 1.
\begin{figure}[h]
    \begin{lstlisting}[language=Javascript]
function test(a){
    return a
}
        \end{lstlisting}
    \caption{Javascript example of very simple function}
    \Description{Javascript example of very simple function}
    \label{fig:simple-func1}
\end{figure}
So does Figure \RefFig{fig:simple-func2}.
\begin{figure}[h]
    \begin{lstlisting}[language=Javascript]
function test(a){
    let b = a
    b = a*b
    b*=42
    return b
}
        \end{lstlisting}
    \caption{Javascript example of less simple function}
    \Description{Javascript example of less simple function}
    \label{fig:simple-func2}
\end{figure}
When we add a control statement with 2 paths our control flow graph now contains 2 possible flows which gives us a CC of 2 as shown in Figure \RefFig{fig:complex-func}.
\begin{figure}[h]
    \begin{lstlisting}[language=Javascript]
function test(a,b){
    if(a>2){
        return a
    }
    return b
}
        \end{lstlisting}
    \caption{Javascript example of more complex function}
    \Description{Javascript example of more complex function}
    \label{fig:complex-func}
\end{figure}
\newline
CC is computed by evaluating the Control Flow Graph (CFG) of a program.
\newline
It can be calculated as
\newline
\begin{verbatim}
    CC = E - N + 2p
\end{verbatim}
Where E = Edges , N = Nodes and p = connected components (which is always one for an independent function) \cite{cycloMaticComplexity}. Therefore the code in Figure \RefFig{fig:cfgcode} is:
\begin{verbatim}
    CC = 4 - 4 + 2
    CC = 2
\end{verbatim}
Which we can also see from following the graph itself.
\begin{figure}[h]
    \begin{lstlisting}[language=Javascript]
function test(a,b){
    if (a < 10){
        a++;
    }else{
        a--;
    }
    return a
}
                \end{lstlisting}
    \caption{Simple Javascript function with control statements}
    \Description{Simple Javascript function with control statements}
    \label{fig:cfgcode}
\end{figure}
\begin{figure}[h]
    \includegraphics[width=.2\textwidth]{appendix/A/AppendixA.png}
    \caption{Control Flow Graph from code sample in \RefFig{fig:cfgcode} created using code2flow \cite{code2flow} see Appendix A}
    \Description{Control Flow Graph of code }
    \label{fig:cfg}
\end{figure}

CC can be an excellent measure to follow as not only does it make us segment our code for readability and extendability it ensures there are not too many test cases for a function.
\newline
McCabe suggested this in his 1976 paper \cite{cycloMaticComplexity}
\begin{verbatim}
"Programmers have been required to calculate complexity 
as they create software modules. When the complexity 
exceeded 10 they had to either recognize and modularize 
subfunctions or redo the software. The intention was to 
keep the "size" of the modules manageable and allow for 
testing all the independent paths..."
\end{verbatim}
Since it's publication there have been various critiques of the metric, the main one being it's correlation with
Source Lines Of Code. On a given piece of code SLOC will linearly track CC, proven in a study of over 1.2 million files of source code \cite{graylin2009cyclomatic}.
While this being the case, the study also found that although LOC has massive predictive power to follow CC in a source code file, there are outliers
which caused variance within the study.
The relationship between CC and SLOC has been further analysed in a 2014 study, the study posits that this variation becomes more apparent
when SLOC increase, control flow statements do no linearly increase with SLOC \cite{CCANDSLOC}.
\begin{verbatim}
    "...there is no evidence of a strong
    linear correlation between SLOC and CC in this large 
    corpus. Suggesting that for Java methods CC measures 
    a different aspect of source code than SLOC..."
    \end{verbatim}
\cite{CCANDSLOC}
Also as CC describes the amount of tests that a piece of software requires it still has it's use in Code Quality analysis.


\subsubsection{\textbf{Halstead Complexity}}
In his 1977 book M.H. Halstead described a set of complexity measures. \cite{HalsteadComplexity}
\newline
These are described as such for any software program.
\begin{itemize}
    \item n\textsuperscript{1} : the number of unique operators
    \item n\textsuperscript{2} : the number of unique operands
    \item N\textsuperscript{1} : the total number of operators
    \item N\textsuperscript{2} : the total number of operands
\end{itemize}
Then several measures can be calculated from these.
\begin{itemize}
    \item Program Vocabulary    : n = n\textsuperscript{1} + n\textsuperscript{2}
    \item Program length        : N = N\textsuperscript{1} + N\textsuperscript{2}
    \item Calculated estimated program length : \newline \^{N} = n\textsuperscript{1} log \textsubscript{2} n\textsuperscript{1} + n\textsuperscript{2} log \textsubscript{2} n\textsuperscript{2}
    \item Volume                : V = N * log \textsubscript{2} n
    \item Difficulty            : D =  $\textfrac{n\textsuperscript{1}}{2}$ * $\textfrac{N\textsuperscript{1}}{n\textsuperscript{1}}$
    \item Effort                : D * V
    \item Time required to program : T = $\textfrac{E}{18}$
    \item Number of delivered bugs : B = $\textfrac{V}{3000}$
\end{itemize}

Example taken from \textit{Eindhoven University of Technology} \cite{softwareEvolution}
\begin{verbatim}
        main()
        {
            int a, b, c, avg;
            scanf("%d %d %d", &a, &b, &c);
            avg = (a+b+c)/3;
            printf("avg = %d", avg);
        }
    \end{verbatim}
In this example c program
\begin{itemize}
    \item n\textsuperscript{1} : 12
    \item n\textsuperscript{2} : 7
    \item n : 19
    \item N\textsuperscript{1} : 27
    \item N\textsuperscript{2} : 15
    \item N : 42
    \item \^{N} : 12 log \textsubscript{2} 12 + 7 log \textsubscript{2} 7 = 62.67
    \item V : 42 * log \textsubscript{2} 19 = 178.4
    \item D : $\textfrac{12}{2}$ * $\textfrac{15}{7}$ = 12.85
    \item E : 12.85 * 178.4 = 2292.44
    \item T : $\textfrac{2292.44}{18}$ = 127.357 seconds
    \item B : $\textfrac{178.4}{3000}$ = 0.059
\end{itemize}


\subsubsection{\textbf{Chidamber and Kemerer Class Design Metrics}}
A metric used in languages with OOP features is the Chidamber and Kemerer metrics \cite{classDesignMetrics}.
These metrics are used to determine the complexity of classes and their methods.

\textbf{Weighted Methods per Class (WMC)}
This describes the number of methods per class, a high WMC count has been found to lead to more problems in the code.
Classes with a high WMC could point to being needed to be split into multiple smaller classes.

\textbf{Depth of Inheritance Tree (DIT)}
This describes the depth of the inheritance of a class.
\begin{figure}[h]
    \begin{lstlisting}[language=Javascript]
class animal{
    constructor(){
        this.isAnimal = true;
    }
}

class dog extends animal{
    constructor(){
        super()
        this.isDog = true;
    }
}

class pug extends dog{
    constructor(){
        super()
         
    }
}
                \end{lstlisting}
    \caption{Example JavaScript classes with inheritance}
    \Description{Example JavaScript classes with inheritance}
    \label{fig:DIT}
\end{figure}

In Figure \RefFig{fig:DIT} we have a few example classes, the DIT values for these classes are as such.

\begin{verbatim}
    animal = 0
    dog = 1
    pug = 2
\end{verbatim}

It is recommended to have a DIT of 5 or less. \cite{classDesignMetrics}

\subsubsection{\textbf{Other Metrics}}
There are other metrics described Generally as Code Smells as described in Clean Code: A Handbook of Agile Software Craftsmanship \cite{cleanCode}.

\textbf{Function Lines}
This is a metric that is tied to the number of lines that can fit on a screen, it is suggestive of bad function design if a function cannot
fit on the screen, with the typical number of lines used being 20.

\textbf{Function Parameters}
A function with too many parameters suggests the need for the function to be split into smaller functions or the use of a parameter object
to allow for parameter specification in the function call.
\begin{verbatim}
"The ideal number of arguments for a function is
zero (niladic). Next comes one (monadic), followed
closely by two (dyadic). Three arguments (triadic)
should be avoided where possible. More than three
(polyadic) requires very special justification—and
then shouldn’t be used anyway"
\end{verbatim} \cite{cleanCode}

\textbf{Function / Class Redefinition}
This is quite self explanatory, it is usually bad practice to define the same function and/or class twice.

\textbf{Self Inheritance}
Another self explanatory measure, a class inheriting from itself would be redundant and possibly syntax breaking in many languages.


\subsubsection{\textbf{Problems with metrics}}
While considering these metrics it is important to remember the problems associated with them. Not only the critiques (as previously discussed) 
but also that written software source code is hard to purely quantify. 
Therefore it is important to ensure that metrics are used as an indication of a possible problem and not as evidence of a 
problem in it's entirety.
\newline
\begin{figure}[h]
    \includegraphics[width=.4\textwidth]{images/goodhart.png}
    \caption{Goodhart's Law Comic from Sketch Plantations \cite{sketchplanations}}
    \Description{Goodhart's Law Comic from Sketch Plantations  }
    \label{fig:goodhart}
\end{figure}
One way to understand this is Goodhart's law \cite{Goodhart1984}, first described to say:
\begin{verbatim}
    "Any observed statistical regularity 
    will tend to collapse once pressure is 
    placed upon it for control purposes"
\end{verbatim}
\cite{Goodhart1984}
Which was generalised by Marilyn Strathern to be:
\begin{verbatim}
    "When a measure becomes a target, 
    it ceases to be a good measure"
\end{verbatim}
\cite{strathern1997improving}
This is to say that when we use a statistic or a metric as the pure evaluation of a system, the person or organisation trying to fit the metric will 
seek to improve their score on the metric to the detriment of the overall system as emblematized by Figure \RefFig{fig:goodhart}.
\newline
While Goodhart used this to describe the Monetary Policy in the United Kingdom, it has been used in various situations such as 
Automotive Carbon Emissions, Individual Productivity in Software Engineering and Ratings in the British University system 
just to name a few \cite{NBERw22911} \cite{FRITZ201667} \cite{strathern1997improving}.


\subsubsection{\textbf{Social and ethical consdierations with research}}
A 
\newline problem with the research into metrics on code quality is the collection of source code to analyse, the data collected has 2 primary sources, 
Companies providing their software to researchers and use of Open Sourced Repositories. Both of these have problems associated with them. 
Companies have an incentive to want their software to be seen as superior to their competitors, this means that a company is likely to share cherry picked code 
that that consider to be of good quality, skewing the data. Open source repositories have a different and more 
ethical concern, is that participation in Open Source communities is skewed towards the perspectives of Male programmers in the age category of 
25-34 \cite{openSourcePerspective} \cite{womenInOpenSource}.
\begin{figure}[h]
    \includegraphics[width=.4\textwidth]{images/Age-distribution-for-open-source-and-general-developers.png}
    \caption{Graph showing Differing ages between \newline Open Source Developers and Employed Developers \cite{openSourcePerspective}}
    \Description{Graph showing Differing ages between Open Source Developers and Employed Developers}
    \label{fig:ageOpenSource}
\end{figure}

\subsection{Existing Code Quality Tools}
There are many tools used by the modern developer to increase the quality of their code, as shown these are important to be used in industry, but should 
also be used in education \cite{codeQualityEducation} \cite{codeQualityStudents}.
\subsubsection{\textbf{AST Explorer}}
AST explorer is a web application that allows you to view the JSON representation of the Abstract Syntax Tree (AST) of a language, it supports over 20 languages and 
up to 20 parsers for a given language. It also has the ability depending on the parser used to suggest readability and code styling fixes \cite{astexplorer}.
\subsubsection{\textbf{Clang Suite}}
Clang is a compiler capable of compiling c++ code, it has various tools associated with it for ensuring code quality.
One of the most popular and widely used throughout industry is Clang Tidy \cite{clangTidy}. Clang Tidy is a command line tool that can 
check for finding typical programming errors and readability, it also can use Clang Static Analyzer \cite{clangStatic} to preform control 
flow analysis.
\subsubsection{\textbf{ESLint}}
ESLint is a static analyser that can be built into an Integrated development environment \cite{wiki:Integrated_development_environment} 
or used as part of a Continuous Integration pipeline \cite{wiki:Continuous_integration}. The problems that ESLint looks for can be customised and 
it can be configured to automatically fix issues it encounters \cite{eslint}.


\subsection{Parsing}
In order to perform static code quality analysis we must parse the language, this is similar to a compiler in the beginning stages 
as shown in Figure \RefFig{fig:codeanalyser} and Figure \RefFig{fig:compiler}.
\begin{figure}[h]
    \includegraphics[width=.11\textwidth]{appendix/B/CodeAnalyser.png}
    \caption{Flow Diagram of stages of code analyser See Appendix B}
    \Description{Flow Diagram of stages of code analyser}
    \label{fig:codeanalyser}
\end{figure}
\begin{figure}[h]
    \includegraphics[width=.11\textwidth]{appendix/C/compiler.png}
    \caption{Flow Diagram of stages of a compiler See Appendix A}
    \Description{Flow Diagram of stages of a compiler}
    \label{fig:compiler}
\end{figure}
A parser takes input data, in our case Source Code, as text and builds a Data Structure, in our case an Abstract Syntax Tree (ABS) that represents 
the structure of the input. 
\subsubsection{\textbf{Automatic Parsing}}
A parser can be written by hand or can be generated by a parser generator, this is what we call Automatic Parsing or Semi-Automatic Parsing. 
An example of a Parser Generator is Nearley.js\cite{nearley.js} which implements the Earley Parsing Algorithm \cite{earley}.
\label{section:bnf}
Parser Generators typically take a definition of the language, the most common format of which is Backus-Naur-Form (BNF). 
BNF is what we call a meta language 
this is because it is a language that describes the syntax of another language, in this case a programming language. In the case of the example shown in Figure \RefFig{fig:bnf-emca} 
a "CommonToken" can be many different things, of these it can be an "IdentifierName" which can either be an "IdentifierStart" or two tokens together in the form of "IdentifierName" and "IdentifierPart"
\cite{bnf}.
\begin{figure}[h]
\begin{verbatim}
CommonToken::
    IdentifierName
    Punctuator
    NumericLiteral
    StringLiteral
    Template
        
IdentifierName::
    IdentifierStart
    IdentifierName IdentifierPart
        \end{verbatim}
    \caption{BNF grammar example in the form of an Excerpt from the ecmascript grammar definition \cite{ecmascript2017}}
    \Description{BNF grammar example in the form of an Excerpt from the ecmascript grammar definition}
    \label{fig:bnf-emca}
\end{figure}
\subsubsection{\textbf{Manual Parsing}}
If we don't want to use a parser generator we can write the parser ourselves, there are a few distinctions between the different 
methods of parsing that it is important to be aware of.

\paragraph{Top-down vs Bottom-up. } A Top-Down parser works from the top of the parse tree and works down to the lowest element (or leaf). A Bottom-Up Parser is the opposite of this 
which starts from the lowest leaf and works it's way up to the top of the parse tree. Bottom-up parsers tends to use right most derivation and Top-down parsers tend to use left most derivation, meaning that when evaluating nodes the right or left child of the current node is chosen. 
The difference is shown in a top down parsing manner between left and right most derivation in Figure \RefFig{fig:leftdir} and Figure \RefFig{fig:rightdir} \cite{topDownBottomUp}. 
\begin{figure}[h]
    \includegraphics[width=.4\textwidth]{appendix/L/Leftmostderivation.png}
    \caption{Top Down Left Most Derivation of Parse Tree Diagram See Appendix L}
    \Description{Top Down Left Most Derivation of Parse Tree Diagram See Appendix L}
    \label{fig:leftdir}
\end{figure}
\begin{figure}[h]
    \includegraphics[width=.4\textwidth]{appendix/M/RightMostDerivation.png}
    \caption{Top Down Right Most Derivation of Parse Tree Diagram See Appendix M}
    \Description{Top Down Right Most Derivation of Parse Tree Diagram See Appendix M}
    \label{fig:rightdir}
\end{figure}

\paragraph{Recursive Descent.}
In section \RefFig{section:bnf} we discussed the grammar of a parser, a Recursive Descent Parser mimics this grammar in it's implementation , 
this will be shown later on in the report in the development section.
\paragraph{Predictive Parsing.} Another feature of parsing is Backtracking, in certain parsers , possibilities for the correct parse are evaluated and abandoned. Although with 
smart creation of grammar it is possible to use a lookahead to the next token to predict what to parse \cite{practicalCompiler}.


\subsubsection{Abstract Syntax Tree.}
\begin{figure}[h]
    \includegraphics[width=.2\textwidth]{images/abstract-syntax-tree.png}
    \caption{Abstract Syntax Tree of Euclidean Algorithm \cite{astwiki}}
    \Description{graphical representation of abs of euclid algorithm}
    \label{fig:abs}
\end{figure}
Whatever parsing methodology we use, we need to create an AST this in an Abstract Syntax Tree. We can then use this representation of the code to preform our analysis.
The below code has been transformed into the AST in Figure \RefFig{fig:abs}
\begin{figure}[h]
    \begin{lstlisting}[language=Javascript]
        while(b != 0){
            if(a > b){
                a = a - b
            }else{
                b = b -a 
            }
        }
        return a
        \end{lstlisting}
    \caption{Javascript example of euclid's algorithm}
    \Description{Javascript example of euclid's algorithm}
    \label{fig:euclid}
\end{figure}

\subsection{Legal issues relating to Software Quality}
As the world of software development progresses and the measurement of code quality becomes standard practice 
ensuring high quality software can become a criminal liability. It could be argued that it is negligence to not use 
such a common practice as Code Quality analysis in the checking for bugs. Under the Tort Theory of Negligence. \cite{legalLiability}

\begin{verbatim}
"...responsibility is limited to only harmful defects
that could have been detected and corrected through 
“reasonable” software practices." 
\end{verbatim} \cite{legalLiability}.
Therefore if a defect in the software was to cause harm to someone, and that bug could have been found via the use of Code Quality Analysis 
,the person or organisation could be liable.

\subsection{Summary}
In summary, we have seen that Code Quality Analysis and the tools used to perform it , is and are valuable software development practices. Not only in 
the monetary sense but in the learning it can convey and facilitate. We have seen the types of metrics Code Quality can measure and discussed the 
critiques and rebuttals, not only to the metrics themselves but to following a metric in itself. We have discussed some of the various tools used 
and the development patterns used to construct them.
\newline
It is now time to implement such a system, to understand it from a deeper level and to produce an artifact that would provide value to end users.





\section{Methodology}

Agile vs Waterfall

Waterfall
Agile

Initial backlog / user stories / user personas


\section{Development}
Development was tracked using a diary See Appendix S.
\newline
We will discuss the Technologies that were used during development and the reasons they were used. We will then take a walk through 
each sprint and discuss the development work done during each.
\subsection{Technologies}
\subsubsection{Front-end}
\paragraph{React.js} \cite{react} Is a front-end application framework, it's benefits include the ability to segment code into 
components that can be reused throughout the application and it's speed in the browser, due to optimisations in it's bundling features. It was chosen 
due to these benefits and the fact the student had previous experience using it. 
There was a possibility of using another framework such as AngularJS \cite{angularjs} but as it is entering end of life in favour of the TypeScript 
version of Angular and the student had little experience with TS it felt more appropriate to use React.js.
\paragraph{Prism.js} \cite{prism} is a javascript highlighting library. It was chosen due to it's various themes and the speed of execution.
\subsubsection{Back-end}
A javascript based backend was chosen to ensure compatibility between the front-end and back-end of the system.
\paragraph{Node.js} \cite{node.js} is a javascript framework, it was chosen for it's massive amount of packages and the students experience in 
using it.
\paragraph{Express.js} \cite{express} is a node backend framework, it was chosen for it's simplicity of hoisting the static files to the 
server.

\subsection{Sprint 1}
The first sprint was interrupted due to the student catching covid. Despite this the student created a baseline for development, creating the basis of the
web application. A React.js\cite{react} front end was created using the React.js tool of Create-react-app \cite{createReactApp}. Simple backend infrastructure was
created, a node.js\cite{node.js} application using express.js\cite{express} to serve the static webpage created from React.js.
\subsection{Sprint 2}
Using a website from my research \cite{astexplorer} as inspiration the student created initial designs as shown in Figures
~\ref{fig:codeeditor1} ~\ref{fig:codeeditor2} ~\ref{fig:codeeditor3} ~\ref{fig:codeeditor4} ~\ref{fig:codeeditor5}
~\ref{fig:codeeditor7}.
These designs were created with the understanding that the code-editor should be as large as possible to allow
the user to edit freely.
\begin{figure}
    \includegraphics[width=.4\textwidth]{appendix/N/code-editor1.png}
    \caption{Code editor designs See Appendix N}
    \Description{Code editor designs See Appendix N}
    \label{fig:codeeditor1}
\end{figure}
\begin{figure}
    \includegraphics[width=.4\textwidth]{appendix/N/code-editor2.png}
    \caption{Code editor designs See Appendix N}
    \Description{Code editor designs See Appendix N}
    \label{fig:codeeditor2}
\end{figure}
\begin{figure}
    \includegraphics[width=.4\textwidth]{appendix/N/code-editor3.png}
    \caption{Code editor designs See Appendix N}
    \Description{Code editor designs See Appendix N}
    \label{fig:codeeditor3}
\end{figure}
\begin{figure}
    \includegraphics[width=.4\textwidth]{appendix/N/code-editor4.png}
    \caption{Code editor designs See Appendix N}
    \Description{Code editor designs See Appendix N}
    \label{fig:codeeditor4}
\end{figure}
\begin{figure}
    \includegraphics[width=.4\textwidth]{appendix/N/code-editor5.png}
    \caption{Code editor designs See Appendix N}
    \Description{Code editor designs See Appendix N}
    \label{fig:codeeditor5}
\end{figure}
\begin{figure}
    \includegraphics[width=.4\textwidth]{appendix/N/code-editor-activity-diagram.png}
    \caption{Code editor activity diagram See Appendix N}
    \Description{Code editor activity diagram See Appendix N}
    \label{fig:codeeditor6}
\end{figure}
\begin{figure}
    \includegraphics[width=.4\textwidth]{appendix/N/wireframe1.png}
    \caption{Code editor wireframe See Appendix N}
    \Description{Code editor wireframe See Appendix N}
    \label{fig:codeeditor7}
\end{figure}

The next step was to highlight the code entered, with the research into prism.js \cite{prism} it seemed perfect for use
in highlighting the code. The only problem is that prism.js is intended for highlighting static code
These items are also described in the User Stories Appendix as seen on the prism.js Examples page. An excellent article was found written by
Oliver Geer \cite{olivergeer} describing the process of having a textarea above a highlighted code element.

\begin{figure}[h]
    \begin{lstlisting}[language=Javascript]
<textarea
    ref={this.state.editingRef}
    placeholder="Enter Code Here "
    className={styles.editing}
    spellCheck="false"
    onChange={(ev) => {
        this.handleInput(ev);
    }}
    onScroll={(ev) => {
        this.handleScroll(ev);
    }}
    onKeyDown={(ev) => {
        this.handleKeyDown(ev);
    }}
></textarea>
<pre
    ref={this.state.highlightingRef}
    className={styles.highlighting}
    aria-hidden="true"
    >
<code className="language-javascript" id="highlighting-content">
    {this.state.text}
</code>
    \end{lstlisting}
    \caption{React code for code editor from "/client/src/components/code\textunderscore{}editor/CodeEditor.jsx" See Appendix K}
    \Description{React code for code editor from "/client/src/components/code\textunderscore{}editor/CodeEditor.jsx" See Appendix K}
    \label{fig:react-codeeditor}
\end{figure}

As shown in Figure \RefFig{fig:react-codeeditor} the textarea has events attached in the react fashion \cite{reactEvents}. Both elements have
React Refs \cite{reactRefs} attached to allow us to access the elements throughout our React Component.
\begin{figure}[h]
    \begin{lstlisting}[language=Javascript]
handleInput(event) {
    event.persist(); // stops react from recycling the SyntheticEvent on re-render
    let text = this.state.editingRef.current.value;
    if (text[text.length - 1] == "\n") {
        text += " ";
    }
    this.setState({ text });
}
handleKeyDown(event) {
    if (event.key === "Tab") {
      event.preventDefault();
      this.handleTab();

      return;
}

handleTab() {
    let text = this.state.text;

    let before = text.slice(0, this.state.editingRef.current.selectionStart);
    let after = text.slice(
      this.state.editingRef.current.selectionEnd,
      text.length
    );
   
    let cursor = this.state.editingRef.current.selectionStart + 1;
    console.log(cursor);
    
    let newText = before + "\t" + after;
    

    this.state.editingRef.current.value = newText;
    this.setState({ text: newText }, () =>this.state.editingRef.current.setSelectionRange(cursor, cursor));
  }
    \end{lstlisting}
    \caption{Event handlers for code editor from "/client/src/components/code\textunderscore{}editor/CodeEditor.jsx" See Appendix K}
    \Description{Event handlers for code editor from "/client/src/components/code\textunderscore{}editor/CodeEditor.jsx" See Appendix K}
    \label{fig:event-handlers}
\end{figure}
When the onChange listener is fired on the textarea, the text from the text is set to state so we can access in our render function. 
We add an extra character to the text as a code block will ignore an empty line as shown in Figure \RefFig{fig:event-handlers}.
\newline
On a key event being fired we check if tab is the key pressed, if it is we cancel the tabbing by calling "event.preventDefault()" \cite{preventDefault}.
We then get the cursor position and insert a tab between before and after the selection. as shown in Figure \RefFig{fig:event-handlers}.

We set the state to the new text, starting the previous process again and then in the callback of setState \cite{reactSetState} 
we update the cursor position.
\newline

This is all tied together by the process described by Figure \RefFig{fig:codeeditor6}. The code set to state is used in the render function and applied 
to the code element as shown in Figure \RefFig{fig:final-steps} by specifying "this.state.text". After the render function completes 
a special react function activates "componentDidUpdate" \cite{reactDidUpdate} which is called by react when the component updates e.g. via a state update. So once we set the text we can call 
highlight on it.

\begin{figure}[h]
    \begin{lstlisting}[language=Javascript]
<code className="language-javascript" id="highlighting-content">
    {this.state.text}
</code>

componentDidUpdate() {
    setTimeout(() => Prism.highlightAll(), 0);
  }

    \end{lstlisting}
    \caption{Final steps of creating highlighted text from  "/client/src/components/code\textunderscore{}editor/CodeEditor.jsx" See Appendix K}
    \Description{Final steps of creating highlighted text from "/client/src/components/code\textunderscore{}editor/CodeEditor.jsx" See Appendix K}
    \label{fig:final-steps}
\end{figure}

\subsection{Sprint 3}
The focus for Sprint 3 was to get the deployment sorted for user testing, there was a problem here. The css for highlighting using prism was not working 
when using a development build. The problem was that the bundler was packaging the scss files as modules, meaning that prism styling wasn't being applied. 
Scss modules are beneficial when using multiple components as you can specify the same class name and have them bundled differently. The fix for this was to 
update the react version as this was a bug fixed in a newer version. An overlay was added to inform the user of the testing procedure.
The code-editor was then deployed to heroku 
\href{https://cq-code-editor-testing.herokuapp.com/}{https://cq-code-editor-testing.herokuapp.com/} and a questionnaire was sent out to potential users.

\subsubsection{Parser Course.}
During this sprint the student realised they were lacking knowledge on the implementation of a parser so the student enrolled themselves in a course 
Building a parser from scratch by Dmitry Soshnikov \cite{parserCourse} which is aimed at teaching the basics of compilers by creating 
a parser for a hypothetical language. The rest of the sprint was used following this course.


\subsection{Sprint 4}
The student completed the course on parsing and started on creating their own version of the parser. As previously described in the background section. This 
would be a \textbf{Top-down Recursive Descent Parser}. Taking the knowledge from the course and expanding on it to create a JavaScript parser.
\begin{figure}[h]
    \begin{lstlisting}[language=Javascript]
/**
* Initial Parse Function
* 
* 
* Starts tokenizer and updates lookahead
* 
* 
* Starts parsing from Program
* 
* Attaches comments if available
* @param {string} input - Input string of source code
* @returns {object} - Program AST 
*/
parse(input) {
    this.source = input;
    this.tokenizer.update(this.source)
    this.lookahead = this.tokenizer.next();
    const ast = this.Program()
    if (this.tokenizer.comments.length) {
        ast.comments = this.tokenizer.comments
    }
    return ast
}
   
/**
* Program:
* 
*      -> StatementList
* @returns {object} Program AST
*/
Program() {
    return {
        type: AST_TYPES.Program,
        body: this.StatementList()
    }
}   
    \end{lstlisting}
    \caption{Initial parser steps  See Appendix O}
    \Description{Initial parser steps  See Appendix O}
    \label{fig:parser1}
\end{figure}
\subsubsection{Tokenising.}
Tokenising is the initial stage of parsing, we need to create tokens for our parser to parse. 
The student utilised a Regular Expression \cite{regexp} array to create a token hierarchy as seen in Figure \RefFig{fig:tokenHierarchy} where regular expressions are 
matched with their corresponding token See TOKEN\textunderscore{}CONST\textunderscore{}TYPES.js in Appendix Q. This is then looped through by the tokeniser and if the token is not 
not found within the hierarchy an error is raised, if it is found it is checked for special cases e.g. the multi line string where the location must 
be handled because of the column and line tracking. Then the token is returned. See Figure \RefFig{fig:tokeniser}.
\begin{figure}[h]
    \begin{lstlisting}[language=Javascript]
        if (type === TOKEN_TYPES.MULTI_LINE_STRING) {
            this.handleMultiLine(tokenValue)
            type = TOKEN_TYPES.STRING; // we've handled the multilines so just treat as a string
        }
        return {
            type: type,
            value: tokenValue,
            loc: { start: pos, end: this.position() },
        }
    }
throw new ParseSyntaxError(`Unexpected token: "${cur[0]}" at ${pos.line}:${pos.column}`, 
        { type: "Unknown", value: cur[0], loc: { start: pos, end: pos } })
    \end{lstlisting}
    \caption{Tokeniser.js  See Appendix Q}
    \Description{Tokeniser.js  See Appendix O}
    \label{fig:tokeniser}
\end{figure}
\begin{figure}[h]
    \begin{lstlisting}[language=Javascript]
const TOKEN_SPEC = [
    [/^\n/, TOKEN_TYPES.NEWLINE], // caught by below, must come before
    [/^\s/, TOKEN_TYPES.WHITESPACE],
    [/^\/\/.*/, TOKEN_TYPES.SINGLE_LINE_COMMENT],
    [/^\/\*[\s\S]*?\*\//, TOKEN_TYPES.MULTI_LINE_COMMENT],
    \end{lstlisting}
    \caption{tokenHierarchy.js  See Appendix Q}
    \Description{tokenHierarchy.js  See Appendix O}
    \label{fig:tokenHierarchy}
\end{figure}
\subsubsection{Parsing.}
In Figure \RefFig{fig:parser1} we see the parse function that starts everything, our grammar starts from the Program node which has the grammar 

\begin{verbatim}
    Program
        StatementList
\end{verbatim}
This again shows the similarity between our final code and the Grammar as described in the background section. In Figure \RefFig{fig:parser2} we parse a 
StatementList which is added to based on the null check performed in Figure \RefFig{fig:parser3} because we eat semicolons and newlines as if they were empty statements 
and don't add them to the parse tree, otherwise a normal statement is added from the grammar definition. We keep following the grammar tree down until we get to 
the most basic of elements which is the Literal as shown in \RefFig{fig:parser4}.
\newline
In each of these examples we can see a few things, the use of the lookahead to determine the type, this is the predictive element of the parser and the use of 
AST\textunderscore{}TYPES which is a constant file used to aid development which specifies the string names of all the AST node types See Appendix P.
\begin{figure}[h]
    \begin{lstlisting}[language=Javascript]
        /**
        * StatementList
        * 
        *      StatementList Statement -> Statement ...
        * @param {string}[d=null] stopLookingPast 
        * @returns {Object[]} Array of Statements
        */
       StatementList(stopLookingPast = null) {
           let list = this.addStatementIfNotNull([])
           while (this.lookahead.type !== TOKEN_TYPES.EOF && this.lookahead.type !== stopLookingPast) {
               list = this.addStatementIfNotNull(list)
           }
           return list
       }
   
       /**
        * Attempt to parse statement and add to list if not null
        * @param {Array} list StatementList
        * @returns {Array} StatementList
        */
       addStatementIfNotNull(list) {
           const statement = this.Statement()
           if (statement != null) {
               list.push(statement)
           }
           return list
       }
    \end{lstlisting}
    \caption{StatementList parsing See Appendix O}
    \Description{StatementList parsing  See Appendix O}
    \label{fig:parser2}
\end{figure}
\begin{figure}[h]
    \begin{lstlisting}[language=Javascript]
        /**
        * Statement 
        * 
        *      : ExpressionStatement
        *      | ForStatement
        *      | DoWhileStatement
        *      | WhileStatement
        *      | ReturnStatement
        *      | ClassDeclaration
        *      | FunctionDeclaration
        *      | VariableStatement
        *      | BlockStatement
        *      | IfStatement       
        *      | null : EOF | SEMI_COLON | NEWLINE
        *      
        * @returns {Object} Statement
        */
       Statement() {
           if (this.lookahead.type === TOKEN_TYPES.EOF) {
               return null
           }
           switch (this.lookahead.type) {
               case TOKEN_TYPES.SEMI_COLON:
                   this.eat(TOKEN_TYPES.SEMI_COLON);
                   return null;
               case TOKEN_TYPES.NEWLINE:
                   this.eat(TOKEN_TYPES.NEWLINE);
                   return null
    \end{lstlisting}
    \caption{Statement parsing See Appendix O}
    \Description{Statement parsing  See Appendix O}
    \label{fig:parser3}
\end{figure}
\begin{figure}[h]
    \begin{lstlisting}[language=Javascript]
        /**
        * Literal propretor 
        * 
        * Literal
        * 
        *      :NumericLiteral
        *      :StringLiteral
        *      :BooleanLiteral
        *      :NullLiteral
        * 
        * @throws {ParseSyntaxError} Throws error on unexpected literal production
        * @returns {Object} Literal 
        */
       Literal() {
           switch (this.lookahead.type) {
               case TOKEN_TYPES.NUMBER:
                   return this.NumericLiteral();
               case TOKEN_TYPES.STRING:
                   return this.StringLiteral();
               case TOKEN_TYPES.TRUE:
                   return this.BooleanLiteral(this.lookahead.type);
               case TOKEN_TYPES.FALSE:
                   return this.BooleanLiteral(this.lookahead.type);
               case TOKEN_TYPES.NULL:
                   return this.NullLiteral();
           }
           /* istanbul ignore next */
           const loc = this.lookahead == null ? "" : `at ${this.lookahead.loc.start.line}:${this.lookahead.loc.start.col}`;
           /* istanbul ignore next */
           const type = this.lookahead == null ? "unknown" : `"${this.lookahead.type}"`;
           /* istanbul ignore next */
           throw new ParseSyntaxError(`Unexpected literal production of type: ${this.lookahead.type} : ${this.lookahead.value} at ${this.lookahead.loc.start.line}:${this.lookahead.loc.start.column}`, this.lookahead)
       }       
    \end{lstlisting}
    \caption{Literal parsing See Appendix O}
    \Description{Literal parsing  See Appendix O}
    \label{fig:parser4}
\end{figure}
\newline
\subsubsection{Testing.} Testing was performed using jest \cite{jest} and used to test every part of the parser. All of the testing suites used a test table which 
which allowed multiple tests to be run in the same suite with similar descriptions. In \RefFig{fig:test} the test checks that the program parses 
an empty statement correctly. "testTable" is an array of arrays which contain ["input",{expected}].
\begin{figure}[h]    
\begin{lstlisting}[language=Javascript]
        import Parser from "../Parser"
        // table of tests [program,expectedOutput]
        const testTable = [
            [
                `;
        `, {
                    "type": "Program",
                    "body": []
                }
            ]
        
        ]
        
        
        
        
        const parser = new Parser()
        describe('Testing empty statement', () => {
            describe.each(testTable)('parsing %s', ((program, expected) => {
        
                test(`returns ${JSON.stringify(expected)}`, () => {
                    expect(parser.parse(program)).toEqual(expected)
                })
            }))
        
        })
    \end{lstlisting}
    \caption{Example test  See Appendix R}
    \Description{Example test  See Appendix R}
    \label{fig:test}
\end{figure}
\subsection{Sprint 5}
Sprint 5 continued with progress on the parser. More javascript support was added, including the ability for statements to be ended by 
newlines, unary and not operators, functions declarations, iteration statements, javascript class definitions, function calls and finally memmber expressions.

There was also AutoComplete and autofill functionality added due to user testing, this will be described in user evaluation.

Finally the rest of focus was used on the creation of the evaluation metrics, this was made quite simple by the use of the created AST.
Cyclomatic Complexity was analysed by using the visitor pattern to crawl through the AST node and increment the complexity when finding a jumping statement, see appendix T /complexity/cyclomatic.js.
Halstead Complexity was analysed by crawling through the entire AST and finding the operands and operators, see Appendix T /complexity/halstead.js. Both of these are tested in Appendix T /complexity/complexity.test.js
\newline
These are both called by evaluate.js which also checks class complexity measures, SLOC errors and function param errors. The tests for this can be found in Appentix T "evaluate.test.js"
\newpage



\section{Description of the final product}

Front end features
\newpage
\section{Appraisal}
\subsection{System Appraisal}
Multiple methods were used to ascertain the completeness of the application these will be discussed in 
relevance to the ends of the system.
\subsubsection{Back-end}
The back-end of the system was appraised partly through user testing as will be discussed in the 
front-end section next. But primarily through the use of unit and integration tests using jest \cite{jest}.
Testing was brought to 99\%+ coverage in all source files as shown by the overall report in Figure \RefFig{fig:coverage} and 
shown in Appendix V.
\begin{figure}[h]
    \includegraphics[width=.5\textwidth]{images/coverage.png}
    \caption{Testing Coverage See Appendix V}
    \Description{Testing Coverage  See Appendix V}
    \label{fig:coverage}
\end{figure}

\subsubsection{Front-end}
The front end of the system was tested via user evaluation.
\newline
The first of these was a questionnaire on the code editor \cite{codeeditortesting} which was accompanied with the 
participant information sheet and informed consent information. The questionnaire asked the users to voice their opinion qualitatively 
on the artifact. The responses were mixed, most liked the dark design of the application, although there were comments 
that the code editor was too simple and needed more features, this is what prompted the implementation of the autocomplete and autofill 
features.
\begin{figure}[h]
    \includegraphics[width=.5\textwidth]{images/responses-code-editor.png}
    \caption{Responses to the code editor testing}
    \Description{Responses to the code editor testing}
    \label{fig:responses}
\end{figure}
\newline
Another deployment was created with the updated features \cite{codeeditortesting2}.
The previous questionnaire prompted users to email the student if they wanted to participate in more testing, there was a 
response and an interview conducted. The user was asked to further use the system as they normally would and to use the 
new autocomplete and autofill features. The user commented that they liked the new features and especially like that 
it autofilled words they had previously entered. This session also unearthed a bug where in large files autocomplete would stop working 
which was fixed. They also mentioned that they wanted to be able to click to autocomplete, this feature did not end up being implemented.
See Appendix W for anonymised interview notes.
\newline
\newline
Finally the system was evaluated by the use of a Questionnaire. The final system was deployed \cite{finalTesting} and users were prompted to evaluate the system quantitatively.
As shown in Figure \RefFig{fig:responses2} users were happy with the analysis output panels with 100\% of responses being Agree or Strongly Agree. 
Sentiment towards the code editor was mixed but remained neutral at worst. 67\% of responses strongly agreed that this application had made them more interested in 
code quality analysis, this is 
\begin{figure}[h]
    \includegraphics[width=.5\textwidth]{images/finaltestingopinion.png}
    \caption{Responses to the final testing}
    \Description{Responses to the final testing}
    \label{fig:responses2}
\end{figure}

\subsection{Appraisal of Work}
I believe that work was generally good on the project. If I was to do this project again I would likely use 
a pre-made parser, this would have made the project so much easier and allowed for more to be accomplished during the project, understanding this 
it was a learning experience that has given the student deeper understanding of how parsing works.
\newline
I would have also spent less time on research and more time on development, again this gave me a deep understanding of the subject matter but more time for 
development would have been beneficial.

\section{Conclusion}
The goal of this project was to understand the code quality analysis process and 
learn how parsing is done, the student believes this has been achieved. The artifact produced 
meets the requirements of 18/30 of the user stories created, with the majority of the stories left 
uncompleted being smaller stories that are made easier by the work that has been completed.
\newline
Coming into this project with no understanding of how static analysis worked and how parsers were 
made it has been a great learning experience that the student believes will benefit them 
in the long term.
how do we take what we have learned forward 
\newline
The final artifact created needs work to be a widely used application but as the results from the 
user testing show, it has created an interest in the participants in code quality analysis. This is justification 
enough to the student that it has been a success.
\section{Future Work}
\begin{figure}
    \includegraphics[width=.2\textwidth]{images/gantt.png}
    \caption{Gantt Sprint 1}
    \Description{gantt for sprint 1}
    \label{fig:gantt}
  \end{figure}
  Reflecting on the progress so far I believe that I have done a good amount of research which I will now apply to coding, I wish I had started prototying earlier as that would have made the planning much easier to estimate time.
  \newline
  As I am following agile I have made simple plans for sprint 1 where my sprints will be 2 weeks, after the end of sprint I will have a retrospective and review the progress completed in sprint 1.
  \newline
  I believe over the course of the project I should be able to complete every user story following this model.
  \newline
  The biggest pitfall that should be encountered is ensuring the abstract syntax tree is correct is this is the bed that everything else lies on, to ensure this it will be developed with a massive amount of unit and integration testing.


  javscript full syntax

  allow settings 

  cyclomatic complexity extended to logical statements 





%%
%% The acknowledgments section is defined using the "acks" environment
%% (and NOT an unnumbered section). This ensures the proper
%% identification of the section in the article metadata, and the
%% consistent spelling of the heading.
% \begin{acks}
% To Robert, for the bagels and explaining CMYK and color spaces.
% \end{acks}

%%
%% The next two lines define the bibliography style to be used, and
%% the bibliography file.
\bibliographystyle{ACM-Reference-Format}
\bibliography{references}

%%
%% If your work has an appendix, this is the place to put it.
\appendix


\section{First Appendix Section}

Appendix
\subsection{User Stories}



\end{document}
\endinput
%%
%% End of file `sample-sigconf.tex'.
